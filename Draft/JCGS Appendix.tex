%&latex
\documentclass[12pt]{article}
\usepackage{amsmath}
\usepackage{graphicx,psfrag,epsf}
\usepackage{enumerate}
%\usepackage{natbib}
\usepackage{url} % not crucial - just used below for the URL

%\pdfminorversion=4
% NOTE: To produce blinded version, replace "0" with "1" below.
\newcommand{\blind}{0}

% DON'T change margins - should be 1 inch all around.
\addtolength{\oddsidemargin}{-.5in}%
\addtolength{\evensidemargin}{-.5in}%
\addtolength{\textwidth}{1in}%
\addtolength{\textheight}{-.3in}%
\addtolength{\topmargin}{-.8in}%

\providecommand{\mb}[1]{\boldsymbol{#1}}
\newcommand{\bx}{\mb{x}}
\newcommand{\by}{\mb{y}}
\newcommand{\bX}{\mb{X}}
\newcommand{\bY}{\mb{Y}}

\let\oldref\ref
\renewcommand{\ref}[1]{(\oldref{#1})}

%\newcommand{\T}{^{\ensuremath{\mathsf{T}}}} % transpose
\newcommand{\T}{^{\ensuremath{\mathsf{T}}}}           % transpose
\newcommand{\mrsid}{{\sc \texttt{Mr}.~\texttt{Sid}}}
\newcommand{\argmax}{\operatornamewithlimits{argmax}}
\newcommand{\argmin}{\operatornamewithlimits{argmin}}
\newcommand{\diag}{\operatornamewithlimits{diag}}



\begin{document}

\title{Appendix}

\section*{Appendix 1: Standard Kalman Filter Smoother}
\label{sec:appendix1}

\begin{tabular}{l}
\hline
\textbf{Algorithm } Standard Kalman Filter Smoother for estimating the moments \\
$\qquad\quad\quad$ required in the E-step of an EM algorithm for a linear dynamical system\\
\hline
0. Define $\bx_t^{\tau}$ = E($\bx_t|\bY_1^{\tau}$),$\mathbf{V}_t^{\tau}=\text{Var}(\bx_t|\bY_1^{\tau})$, $\hat{\bx}_t \equiv \bx_t^T$ and $\hat{P}_t\equiv V_t^T+\bx_t^T{\bx_t^T}^{\T}$\\
1. Forward Recursions:\\
\hspace{4 mm} $\bx_t^{t-1}=A\bx_{t-1}^{t-1}$\\
\hspace{4 mm} $\mathbf{V}_t^{t-1}=A\mathbf{V}_{t-1}^{t-1}+\mathbf{Q}$\\
\hspace{4 mm} $K_t=\mathbf{V}_t^{t-1}C^{\T}(CV_t^{t-1}C^{\T}+R)^{-1}$\\
\hspace{4 mm} $\bx_t^t$ = $\bx_t^{t-1} + K_t (\by_t - C\bx_t^{t-1})$\\
\hspace{4 mm} $V_t^t=V_t^{t-1}-K_tCV_t^{t-1}$\\
\hspace{4 mm} $\bx_1^0=\mathbf{\pi}_0$, $V_1^0=\mathbf{V}_0$\\
2. Backward Recursions:\\
\hspace{4 mm} $J_{t-1} = V_{t-1}^{t-1}A^{\T}(V_t^{t-1})^{-1}$\\
\hspace{4 mm} $\bx_{t-1}^T=\bx_{t-1}^{t-1}+J_{t-1}(\mathbf{x_t^T-A\bx_{t-1}^{t-1}})$\\
\hspace{4 mm} $V_{t-1}^T = V_{t-1}^{t-1}+J_{t-1}(V_t^T-V_t^{t-1})J_{t-1}^{\T}$\\
\hspace{4 mm} $\hat{P}_{t,t-1}\equiv V_{t,t-1}^T+\bx_t^T{\bx_t^T}^{\T}$\\
\hspace{4 mm} $V_{T,T-1}^T=(I-K_TC)AV_{T-1}^{T-1}$\\
\hline
\end{tabular}

\section*{Appendix 2: Optimizing Over $C$ Matrix}
Terms relevant to $C$ are
\begin{equation}\label{eq:penaltylik1}
f_{\lambda_2}(C;\bX,\bY) = \sum\limits_{t=1}^{T}\left(\frac{1}{2}[\by_t-C\bx_t]^{\T}R^{-1}[\by_t-C\bx_t]\right)+\lambda_2 \|C\|_2.
\end{equation}

In $f_{\lambda_2}(C;\bX,\bY)$, $C$ is a matrix, we vectorized it to ease optimization and notation. Without loss of generality, assume $R$ is the identity matrix in equation \ref{eq:penaltylik1}; otherwise, one can always write equation \ref{eq:penaltylik1} as
\begin{equation*}
\sum\limits_{t=1}^{T}\left(\frac{1}{2}[R^{-\frac{1}{2}}y_t-R^{-\frac{1}{2}}Cx_t]^{\T}[R^{-\frac{1}{2}y_t}-R^{-\frac{1}{2}}Cx_t]\right) + \lambda_2 \|R^{-\frac{1}{2}}C\|
\end{equation*}
Let $\bY' = (y_{11},\ldots,y_{T1},y_{12},\ldots,y_{T2},\ldots,y_{1p},\ldots,y_{Tp})^{\T}$
%\begin{equation*}
%\bY' = (y_{11},\ldots,y_{T1},y_{12},\ldots,y_{T2},\ldots,y_{1p},\ldots,y_{Tp})^{\T}
%\end{equation*}
be a $Tp\times 1$ vector from rearranging  $\bY$. In addition, let
\[
\bX' = \begin{pmatrix}
\bX^{\T}&&\\
&\ddots&\\
&&\bX^{\T}
\end{pmatrix}_{pT\times pd}.
\]
% where $W=\biggl(\bx_1,\ldots,\bx_{T}\biggr)$.
Finally, vectorize $C^{\text{old}}$ as
\begin{equation}\label{eq:vectorizec}
\mathbf{c}^{\text{old}} = (C_{11}^{\text{old}},\ldots,C_{1d}^{\text{old}},C_{21}^{\text{old}},\ldots,C_{2d}^{\text{old}},C_{p1}^{\text{old}},\ldots,C_{pd}^{\text{old}})^{\T}
\end{equation}
where $C_{ij}$ is the element at row $i$ and column $j$ of $C$. With these new notations, the equation \ref{eq:penaltylik1} is equivalent to
\begin{equation}\label{eq:penaltylik11}
f_{\lambda_2}(C;\bX,\bY) = \|\bY'  - \mathbf{X' c}\|_2^2 + \lambda_2\|\mathbf{c}\|_2^2.
\end{equation}
With the Tikhonov regularization, equation \ref{eq:penaltylik11} has closed form solution
\begin{equation}\label{eq:updatec}
\begin{aligned}
\mathbf{c}^{\text{new}} &= (\bX'^{\T}\bX' + \lambda_2\mathbf{I})^{-1}\bX'^{\T}\bY'\\
C^{\text{new}} &=\text{Rearrange } \mathbf{c}^{\text{new}} \text{ by equation }\ref{eq:vectorizec}
\end{aligned}
\end{equation}

\section*{Appendix 3: FISTA Algorithm}
\label{sec:appendix2}
In general, FISTA optimize a target function
\begin{equation}\label{eqn: fistatarget}
\min_{\substack{x\in \mathcal{X}}}\quad \mathbf{F(x;\lambda)} = \mathbf{g(x)}+ \mathbf{\lambda \|x\|_1}
\end{equation}
where $\mathbf{g}: R^n \rightarrow R $ is a continuously differentiable convex function and $\lambda > 0$ is the regularization parameter. A FISTA algorithm with constant step is detailed below\\

%\begin{center}
\begin{tabular}{l}
\hline
\textbf{Algorithm } FISTA$(\mathbf{g},\lambda)$.\\
\hline
 1. Input an initial guess $\mathbf{x_0}$ and Lipschitz constant $\mathbf{L}$ for $\mathbf{\nabla g}$, set $\mathbf{y_1} = \mathbf{x_0},t_1 = 1$\\
 2. Choose $\tau \in (0,1/\mathbf{L}]$; Set $k \leftarrow$ 0.\\
 3. \textbf{loop}\\
 4. \hspace{10mm}		Evaluate $\mathbf{\nabla g(y_k)}$\\
 5.	\hspace{10mm}	Compute $\mathbf{x_{1}}$= $\mathbf{S_{\tau\lambda}(y_k - \tau\nabla g(y_k))}$\\
 6.	\hspace{10mm}	Compute $t_{k+1} = \frac{1+\sqrt{1 + 4 t_k^2}}{2}$\\
 7.	\hspace{10mm}	$\mathbf{y_{k+1}} = \mathbf{x_k} + \left(\frac{t_k - 1}{t_{k+1}})\right (\mathbf{x_k}-\mathbf{x_{k-1}})$\\
 8.	\hspace{10mm}	Set $k \leftarrow k+1$ \\
 9. \textbf{end loop}\\
\hline
\end{tabular}
%\end{center}


\vspace*{10mm}
In the above
\[
\mathbf{S_\lambda (y) = (|y|-\lambda)_{+}\textbf{sign}(y)}=\left\{
\begin{array}{l l }
 y - \lambda & \text{if   } y > \lambda\\
 y + \lambda & \text{if   } y < -\lambda\\
 0 & \text{if   } |y| \leq \lambda .
\end{array}
\right.
\]

The Lipschitz constant $L$ for $\nabla\mathbf{g(z)}=\mathbf{Z}^{\T}(\mathbf{Z}\mathbf{a} -\mathbf{z})$, where $\mathbf{g}(\mathbf{z})=\|\mathbf{Z}^{\T}\mathbf{a} -\mathbf{z}\|_2^2$, is calculated as follows. Denote $\|Z\|$ as the induced norm of matrix $Z$, then $L$ is
\[
L = \sup_{\substack{x\neq y}}\frac{\|\mathbf{Z}^{\T}(\mathbf{Z}x- \mathbf{Z}y)\|}{\|x-y\|}=\sup_{\substack{x\neq 0}}\frac{\|\mathbf{Z}^{\T}\mathbf{Z}x\|}{\|x\|}\leq\|\mathbf{Z}^{\T}\|\|\mathbf{Z}\| = \|Z^{\T}\|\|Z\|.
\]


\section*{Appendix 4: $k$-step predictions with PCA and \mrsid}
\label{sec:appendix3}
\begin{tabular}{l}
\hline
\textbf{Algorithm } $k$-step predictions with PCA and \mrsid\\
\hline
 1. Denote estimations with PCA and \mrsid~as $A_{pca},C_{pca},A_{plds},$ and $C_{plds}$ respectively.\\
 2. PCA estimated latent states at $t=1000$: $x_{1000,pca} = $ column 1000 of $\bX_{d\times T}$ from Section 3.3 \\
 3. \mrsid~estimated latent states at $t=1000$: $x_{1000,pls}$ is from the E step in Section 3.4\\
 4. \textbf{for i = 1 to k}\\
 5. \hspace{10mm}		$x_{1000+k,pca}\ =\ A_{pca}\ x_{999+k,pca}$\\
 6.	\hspace{10mm}	$y_{1000+k,pca}\ =\ C_{pca}\ x_{1000+k,pca}$\\
 7.	\hspace{10mm}	$x_{1000+k,plds}\ =\ A_{plds}\ x_{999+k,plds}$\\
 8.	\hspace{10mm}	$y_{1000+k,plds}\ =\ C_{plds}\ x_{1000+k,plds}$\\
 9. \textbf{end}\\
\hline
\end{tabular}

\section*{Appendix 5: Simulation Data Generation}
\label{sec:appendix4}
\begin{tabular}{l}
\hline
\textbf{Algorithm } Simulation Data Generation\\
\hline
 1. Denote the dimensions as $p$, $d$ and $T$ respectively\\
 2. Generate a $p\times d$ matrix $C_0$ from a standard Gaussian distribution\\
 3. Sort each column of $C_0$ in ascending order to get matrix $C$ \\
 4. Generate a $d\times d$ matrix $A_0$ from a standard Gaussian distribution\\
 5. Add a multiple of the identity matrix to $A_0$\\
 6.	Replace entries in $A_0$ with small absolute values with $0$\\
 7.	Scale $A_0$ to make sure its eigen values are between $-1$ and $1$; use $A_0$ as the A matrix\\
 8.	Let $R$ be a diagonal matrix with positive diagonal entries and $Q$ be the identity matrix \\
 9. Generate simulation data with $A, C, Q$ and $R$ \\
 10. \textbf{end}\\
\hline
\end{tabular}

\end{document}
